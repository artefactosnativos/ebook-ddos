

Yevgeny Zamyatin fue un escritor ruso nacido en Lebedyan, Rusia el 18 de enero de 1884. Fue uno de los primeros escritores soviéticos y fue una figura clave en la literatura rusa del siglo XX. Se graduó de la Escuela Naval de San Petersburgo en 1906 y desarrolló su carrera como ingeniero naval. Durante la Revolución de 1917, fue un oficial del Ejército Rojo.

Zamyatin escribió su novela más conocida, We, entre 1921 y 1922. La obra fue prohibida en la Unión Soviética pero fue publicada en el extranjero. Después de la publicación de la novela, Zamyatin fue arrestado y privado de su derecho a publicar. En 1931, partió a París, donde permaneció hasta su muerte en 1937.

Durante su exilio, Zamyatin escribió varias obras satíricas que criticaban la Unión Soviética. Su trabajo influyó a varios escritores posteriores, como George Orwell. Su obra influyó en el género de la ciencia ficción moderna y fue una voz profética sobre el totalitarismo. El legado de Zamyatin seguirá afectando a la literatura rusa y universal durante años.